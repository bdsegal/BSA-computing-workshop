\documentclass{beamer}
\usetheme{Warsaw}
%\usecolortheme{dolphin}

\usepackage{bm, bbm, amsmath, amsfonts}

\title[]{Data Management and Visualization in \textsf{R}}
\subtitle{BSA Computing Workshop}
\author[]{Brian Segal}
\institute[]{Department of Biostatistics \\ University of Michigan}
\date{February 12, 2016}

\begin{document}

\begin{frame}
\titlepage
\end{frame}

\begin{frame}{Overview}
\begin{block}{Data Management}
Typical workflow
\begin{enumerate}
\item Read in
\item Reshape
\item Split, apply, and combine
\end{enumerate}
\end{block}
To do: make diagram of read in $\rightarrow$ reshape $\rightarrow$ split, apply, combine, with $\rightarrow$ visualize at each step
\end{frame}


\begin{frame}
\frametitle{Outline}
\tableofcontents[hideallsubsections]
\end{frame}

\section{Read in}

\begin{frame}{File management}
\end{frame}

\begin{frame}{Medium files}
\end{frame}

\begin{frame}{Large files}
\end{frame}

\begin{frame}{Files too large to fit into memory}

Note: see Kerby Shedden's site, BioConductor, etc.
\end{frame}

\section{Manipulating datasets}
\subsection{Overview}

\begin{frame}{Data processing steps}
\end{frame}

\subsection{Basic manipulations}
\begin{frame}{dplyr}
\end{frame}

\begin{frame}{data.table}
For maximum speed. Very useful, e.g., if bootstrapping column means or standard deviations.
\end{frame}

\end{document}